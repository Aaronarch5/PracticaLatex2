\documentclass[letter,twocolumn]{revtex4}
\usepackage[utf8]{inputenc}

\usepackage{epsfig}
\usepackage{graphicx}
\usepackage{tikz}
\usetikzlibrary{shapes.geometric, arrows}
\graphicspath{ {images/} }
\usepackage{listings}
\usepackage{color}

\usepackage{xcolor}

\definecolor{codegreen}{rgb}{0,0.6,0}
\definecolor{codegray}{rgb}{0.5,0.5,0.5}
\definecolor{codepurple}{rgb}{0.58,0,0.82}
\definecolor{backcolour}{rgb}{0.95,0.95,0.92}

\lstdefinestyle{mystyle}{
    backgroundcolor=\color{backcolour},   
    commentstyle=\color{codegreen},
    keywordstyle=\color{magenta},
    numberstyle=\tiny\color{codegray},
    stringstyle=\color{codepurple},
    basicstyle=\ttfamily\footnotesize,
    breakatwhitespace=false,         
    breaklines=true,                 
    captionpos=b,                    
    keepspaces=true,                 
    numbers=left,                    
    numbersep=5pt,                  
    showspaces=false,                
    showstringspaces=false,
    showtabs=false,                  
    tabsize=2
}

\lstset{style=mystyle}


\begin{document}

\title{Práctica 2: Calculadora de Matrices}
\author{Archundia Bazán Aarón Antonio, Guerrero Velez Eliseo Milton, Hernández Vázquez Cesar Arturo}
\affiliation{Unidad Profesional Interdisciplinaria en Ingenier\'{\i}a y Tecnolog\'{\i}as Avanzadas del I.P.N.\\ Ingeniería Biónica,Programación Orientada a Objetos}

\date{23 de noviembre de 2020}
\maketitle


\section{Planteamiento del problema}

El problema consiste en desarrollar una calculadora de matrices que trabaje con las siguientes operaciones para matrices bidimensionales de cualquier tamaño: 
\begin{itemize}
    \item Suma
    \item Resta
    \item Multiplicación
    \item Transpuesta
\end{itemize}
Donde el tamaño de la matriz sea dado por el usuario y respete las propiedades matrices así como las reglas para cada uno de los métodos antes propuestos.

\section{Propuesta de Solución}
   

\section{Análisis y diseño}

\subsection{Diagrama de flujo}


\clearpage

\section{Implementación y pruebas}
 

\section{Código fuente comentado}

\lstinputlisting[language=C++]{main.cpp}


\section{Conclusiones}





\end{document}
